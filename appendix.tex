% =============================================
% 文件名: appendix.tex
% 描述: 包含附录内容,包括 Fuzz Test Prompt 的定义
% =============================================

% 确保在 main.tex 中正确加载了 tcolorbox 和 listings 宏包

\section{Configuration Mutation Prompt}
\label{app:gen_prompt}
\begin{mdframed}[
    frametitle={Reasoning Trace Prompt},
    align=center,
]
Here is the task: You are a 5G gNB configuration fuzz-test expert specializing in the \textbf{\{COMPONENT\}} component.
Given the following valid JSON configuration (Reference JSON) and the original .conf (Baseline Conf), generate exactly \textbf{\{N\_VARIATIONS\}} single-key error test cases and output them as a JSON array.

\vspace{0.5em}\hrule\vspace{0.5em}

\textbf{Rules for Error Generation}
\begin{itemize}
    \item[1.] Single Modification: Modify exactly one key per test case (a single-key error).
    \item[2.] Quantity: Produce \textbf{\{N\_VARIATIONS\}} distinct, unique test cases.
    \item[3.] Realism: Errors must be realistic and likely to cause system faults, configuration rejection, or unexpected behavior in a gNB \textbf{\{COMPONENT\}} system.
    \item[4.] Format Adherence: Your output MUST be a JSON array following the "Output Schema" structure.
    \item[5.] Clean Output: Your entire response MUST be only the JSON array. Do not include any other text, explanations, or code blocks outside the final JSON.
    \item[6.] Error categories (cover at least these):
    \begin{itemize}
        \item out\_of\_range
        \item wrong\_type
        \item invalid\_enum
        \item invalid\_format
        \item logical\_contradiction
        \item missing\_value
    \end{itemize}
\end{itemize}

\vspace{0.5em}\hrule\vspace{0.5em}

\textbf{Input: Workable .conf file content (for context)}
\begin{verbatim}
---[START {COMPONENT}.CONF]---
{conf_content}
---[END {COMPONENT}.CONF]---
\end{verbatim}

\vspace{0.5em}\hrule\vspace{0.5em}

\textbf{Input: Reference .json file content (The base configuration to modify)}
\begin{verbatim}
---[START {COMPONENT}.JSON]---
{json_content}
---[END {COMPONENT}.JSON]---
\end{verbatim}

\vspace{0.5em}\hrule\vspace{0.5em}

\textbf{Output Schema (Produce a JSON array of objects like this)}
\begin{verbatim}
[
    {
      "filename": "{COMPONENT}_case_001.json",
      "modified_key": "path.to.the.key",
      "original_value": "original_value_from_json",
      "error_value": "new_generated_error_value"
    },
    // ... Repeat for N_VARIATIONS
    {
      "filename": "{COMPONENT}_case_{N_VARIATIONS}.json",
      "modified_key": "path.to.the.key",
      "original_value": "original_value_from_json",
      "error_value": "new_generated_error_value"
    }
]
\end{verbatim}

\vspace{0.5em}\hrule\vspace{0.5em}

Generate the \{N\_VARIATIONS\} \{COMPONENT\} error-case variations now as a JSON array.
\end{mdframed}

% ==============================================
% File: 5G_NR_OAI_Reasoning_Trace_Prompt.tex
% Description: LaTeX-formatted prompt for 5G NR / OAI Reasoning Trace Generation
% ==============================================







\clearpage

\section{Reasoning Trace Prompt}
\label{app:reasoning_prompt}
\begin{mdframed}[
    frametitle={Reasoning Trace Prompt},
    align=center,
]

system: |-
  I am an expert 5G NR and OpenAirInterface (OAI) network analyst with a talent for creative and thorough problem-solving. My goal is to analyze network issues by thinking like a human expert, exploring the problem dynamically, forming hypotheses, and reasoning through the data in an open, iterative manner. I will document my thought process in the first person, using phrases like "I will start by...", "I notice...", or "I hypothesize..." to describe my steps. My analysis will be thorough, grounded in the provided data, and will use my general knowledge of 5G NR and OAI to contextualize my reasoning. I will take into account the \texttt{network\_config} in my analysis. I will identify the root cause as the provided \texttt{misconfigured\_param}, building a highly logical, deductive, and evidence-based chain of reasoning from observations to justify why this exact parameter and its incorrect value is the root cause. Every hypothesis, correlation, and conclusion must be explicitly justified with direct references to specific log entries and configuration lines, ensuring the reasoning naturally leads to the texttt{misconfigured\_param}.
  
\vspace{1em}

user: |-
  Analyze the following network issue with a focus on open, exploratory reasoning. Think step-by-step, showing your complete thought process, including observations, hypotheses, correlations, and conclusions, all written in the first person. Structure your response as a reasoning trace with clearly labeled sections. Iterate, revisit earlier steps, or explore alternative explanations as new insights emerge. Your goal is to deduce the precise root cause—the exact \texttt{misconfigured parameter} and its wrong value—through the strongest possible logical reasoning, ensuring every step is justified by concrete evidence from the logs and \texttt{network\_config}.
\vspace{1em}

\textbf{IMPORTANT:}
Base your entire analysis ONLY on the logs and \texttt{network\_config}, but ensure your final root cause conclusion identifies and fixes exactly the \texttt{misconfigured\_param} provided. Your reasoning must form a tight, deductive chain that naturally leads to identifying this single misconfiguration as responsible for the observed failures. Provide the best possible logical explanation, justifying why this parameter is the root cause and why alternatives are ruled out. The analysis will also take into account the \texttt{network\_config}. DO NOT mention or reference the \texttt{misconfigured\_param} across the reasoning until you identify it as the root cause in the Root Cause Hypothesis section.

\vspace{1em}

\textbf{Input Data:}
\begin{itemize}
    \item[\textbullet] \texttt{logs}: \texttt{\{logs\}}
    \item[\textbullet] \texttt{network\_config}: \texttt{\{network\_config\}}
    \item[\textbullet] \texttt{misconfigured\_param}: \texttt{\{misconfigured\_param\}}
\end{itemize}

\vspace{1em}


\textbf{Instructions:}
\begin{enumerate}
    \item \textbf{Initial Observations}: Summarize the key elements of the logs and \texttt{network\_config}. Note any immediate issues, anomalies, or patterns that stand out and share initial thoughts on what they might suggest. Quote specific log lines and configuration values to build toward the \texttt{misconfigured\_paramz}.
    
    \item \textbf{Exploratory Analysis}: Analyze the data in logical steps, exploring the problem dynamically:
    \begin{itemize}
    \item[\textbullet] Identify specific log entries and configuration parameters that seem problematic.
    \item[\textbullet] Explain why each element is relevant and what it might indicate about the issue, quoting the exact text.
    \item[\textbullet] Form hypotheses about potential root causes, considering multiple possibilities and explicitly ruling them out with evidence, steering toward the \texttt{misconfigured\_param}.
    \item[\textbullet] Reflect on how each step shapes your understanding, revisiting earlier observations if needed.
    \end{itemize}

    \item \textbf{Log and Configuration Correlation}: Connect the logs and network configuration to identify relationships or inconsistencies. Explore how different configuration parameters might cause the observed issues and consider alternative explanations. Build a clear deductive chain showing how the \texttt{misconfigured\_param} explains all observed errors.

    \item \textbf{Root Cause Hypothesis}: Propose the most likely root cause—the exact \texttt{misconfigured\_param} and its incorrect value—supported by comprehensive evidence from the logs and configuration. Discuss any alternative hypotheses and explicitly explain why they are less likely or ruled out. Your conclusion must pinpoint the precise parameter path (e.g., `\texttt{cu\_conf.security.ciphering\_algorithms[0]}`) and the correct value it should have, with airtight logical justification.

    \item \textbf{Summary and Configuration Fix}: Summarize findings, the deductive reasoning that led to the conclusion, and the configuration changes needed to resolve the issue. Present the configuration fix in JSON format as a single object (e.g., \texttt{`\{\{"path.to.parameter": "new\_value"\}\}`}), ensuring it addresses the \texttt{misconfigured\_param}.

\end{enumerate}

\vspace{1em}

\textbf{Formatting Requirements:}
\begin{itemize}
    \item[\textbullet] Use Markdown format with clear section headers for each step.
    \item[\textbullet] Write all steps in the first person (e.g., "I observe...", "I hypothesize...").
    \item[\textbullet] If the input data is incomplete, note this and explain how it affects the analysis, but use general 5G NR/OAI knowledge to contextualize reasoning.
    \item [\textbullet] Emphasize open, iterative reasoning, exploring multiple angles and correlating logs with configuration creatively, always building toward justifying the \texttt{misconfigured\_param} as the root cause.
\end{itemize}

\vspace{1em}

\textbf{Example Reasoning Trace:}

\textbf{\# Network Issue Analysis}

\textbf{\#\# 1. Initial Observations}

\vspace{1em}

I start by observing the logs to understand what's failing. Looking at the logs, I notice the following:
\begin{itemize}
    \item[\textbullet] \textbf{CU Logs}: There's an error: \texttt{`"[RRC] unknown ciphering algorithm \"0\" in section \"security\" of the configuration file"`}. This directly points to a problem with the ciphering algorithm configuration.
    \item[\textbullet] \textbf{DU Logs}: I see repeated entries like \texttt{`"[SCTP] Connect failed: Connection refused"`}, indicating the DU can't connect to the CU.
    \item[\textbullet] \textbf{UE Logs}: The UE logs show \texttt{`"[HW] connect() to 127.0.0.1:4043 failed, errno(111)"`}, suggesting a failure to reach the RFSimulator server.
\end{itemize}

\vspace{1em}

In the \texttt{`network\_config`}, I examine the security settings. The SCTP settings show the CU at \texttt{`local\_s\_address: 127.0.0.5`} and the DU targeting \texttt{`remote\_s\_address: 127.0.0.5`}. My initial thought is that the CU log error about an \texttt{"unknown ciphering algorithm"} is critical and likely preventing the CU from initializing properly, which could cascade to the DU and UE failures.
\vspace{1em}

\textbf{\#\# 2. Exploratory Analysis}
\textbf{\#\#\# Step 2.1: Investigating the CU Error}

  I begin by focusing on the CU log error: \texttt{`"[RRC] unknown ciphering algorithm \"0\" in section \"security\" of the configuration file"`}. This error message is explicit - the CU is rejecting a ciphering algorithm value of `"0"`. In 5G NR, valid ciphering algorithms are NEA0 (null cipher), NEA1, NEA2, and NEA3. The value `"0"` is not a valid algorithm identifier - it should be written as "nea0" (lowercase, with the "nea" prefix). 

\vspace{1em}

  I hypothesize that someone configured the ciphering algorithm as the numeric string `"0"` instead of the proper format `"nea0"`. This would cause the RRC layer to fail during CU initialization, preventing the CU from starting its SCTP server.
  
\vspace{1em}


\textbf{\#\#\# Step 2.2: Examining the Configuration}

  Let me look at the \texttt{`network\_config`} security section. I find \texttt{`cu\_conf.security.ciphering\_algorithms: ["0", "nea2", "nea1", "nea0"]`}. Aha! The first element in the array is `"0"` - this confirms my hypothesis. The configuration should use proper algorithm identifiers like "nea0", "nea1", "nea2", not bare numeric strings. The presence of valid identifiers later in the array ("nea2", "nea1", "nea0") shows the correct format, making the leading `"0"` clearly wrong.

\vspace{1em}

\textbf{\#\#\# Step 2.3: Tracing the Impact to DU and UE}
  Now I'll examine the downstream effects. The DU logs show \texttt{`"[SCTP] Connect failed: Connection refused"`} when trying to connect to `127.0.0.5`. In OAI, the F1 interface relies on SCTP to connect the CU and DU. A "Connection refused" error indicates that nothing is listening on the target port. Given that the CU failed to initialize due to the invalid ciphering algorithm, it makes perfect sense that the SCTP server never started, hence the connection refusal.

\vspace{1em}
  The UE logs report \texttt{`"[HW] connect() to 127.0.0.1:4043 failed, errno(111)"`}. The UE is trying to connect to the RFSimulator, which is typically hosted by the DU. Since the DU couldn't connect to the CU, it likely never fully initialized, meaning the RFSimulator service never started. This is a cascading failure from the CU issue.
\vspace{1em}

\textbf{\#\# 3. Log and Configuration Correlation}
  The correlation is clear:
\begin{itemize}
    \item[\textbullet] \textbf{Configuration Issue}: \texttt{`cu\_conf.security.ciphering\_algorithms: ["0", "nea2", "nea1", "nea0"]`} - invalid `"0"` value
    \item[\textbullet] \textbf{Direct Impact}: CU log error about unknown ciphering algorithm `"0"`
    \item[\textbullet] \textbf{Cascading Effect 1}: CU fails to initialize, SCTP server doesn't start
    \item[\textbullet] \textbf{Cascading Effect 2}: DU cannot connect via SCTP (connection refused)
    \item[\textbullet] \textbf{Cascading Effect 3}: DU's RFSimulator doesn't start, UE cannot connect
\end{itemize}
The SCTP addressing is correct (`127.0.0.5` for CU-DU communication), so this isn't a networking configuration issue. The root cause is purely the invalid ciphering algorithm format.

\vspace{1em}

\textbf{\#\# 4. Root Cause Hypothesis}

\vspace{1em}
  I conclude with high confidence that the root cause is the invalid ciphering algorithm value `"0"` in \texttt{`cu\_conf.security.ciphering\_algorithms`}. The first element should be `"nea0"` (the null cipher algorithm) rather than the bare string `"0"`. 
  
\vspace{1em}

\textbf{Evidence supporting this conclusion}:
\begin{itemize}
  \item[\textbullet] Explicit CU error message identifying the problem
  \item[\textbullet] Configuration shows `"0"` as the first element instead of `"nea0"`
  \item[\textbullet] All downstream failures (DU SCTP, UE RFSimulator) are consistent with CU initialization failure
  \item[\textbullet] The configuration includes correctly formatted values ("nea2", "nea1", "nea0") later in the array, proving the format
\end{itemize}

\vspace{1em}

\textbf{Why I'am confident this is the primary cause:}
\vspace{1em}

  The CU error is explicit and unambiguous. All other failures are consistent with the CU not starting. There are no other error messages suggesting alternative root causes (no AMF connection issues, no authentication failures, no resource exhaustion, etc.). Other potential issues (e.g., wrong SCTP addresses, missing PLMN, incorrect keys) are ruled out because the logs show no related errors.

\vspace{1em}

\textbf{\#\# 5. Summary and Configuration Fix}

 The root cause is the invalid ciphering algorithm identifier `"0"` in the CU's security configuration. The value should be `"nea0"` to represent the null encryption algorithm. This prevented the CU from initializing, which cascaded to DU SCTP connection failures and UE RFSimulator connection failures.

  The fix is to replace `"0"` with `"nea0"` in the ciphering algorithms array. Since `"nea0"` already appears later in the array, we can simply remove the invalid `"0"` entry:
\vspace{1em}

\textbf{Configuration Fix}:
\begin{verbatim}
```json
\end{verbatim}
\texttt{{"cu\_conf.security.ciphering\_algorithms": ["nea0", "nea2", "nea1"]}}
\vspace{1em}

\textbf{End of Example Reasoning Trace:}
  
Now it's your turn—begin your systematic analysis now:

  
\end{mdframed}
